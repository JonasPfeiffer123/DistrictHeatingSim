\section{Einleitung}
Dieses Dokument beschreibt die Berechnungsmethode, die in der Datei \texttt{SanierungsanalysefuerGUI.py} implementiert ist. Die Berechnungen umfassen sowohl die Ermittlung des Wärmebedarfs eines Gebäudes vor und nach einer Sanierung als auch eine wirtschaftliche Analyse der Sanierung, einschließlich Amortisationszeit, Net Present Value (NPV), und Return on Investment (ROI). 

\section{Modellierung des Gebäudes und Berechnung des Wärmebedarfs}
Die \texttt{Building}-Klasse repräsentiert ein Gebäude und berechnet den maximalen Heizwärmebedarf, den jährlichen Heizwärmebedarf sowie den jährlichen Warmwasserbedarf. Diese Berechnungen basieren auf den thermischen Eigenschaften des Gebäudes und den klimatischen Bedingungen (aus TRY-Daten).

\subsection{U-Werte und Wärmeverluste}
Die Wärmedurchgangskoeffizienten (U-Werte) beschreiben die Wärmeverluste durch die verschiedenen Bauteile des Gebäudes (z.B. Wände, Dach, Fenster, Türen, Boden). Der Wärmeverlust pro Kelvin Temperaturdifferenz wird berechnet als:
\[
\text{Wärmeverlust}_\text{Teil} = A_\text{Teil} \times U_\text{Teil}
\]
wobei \( A_\text{Teil} \) die Fläche des jeweiligen Bauteils und \( U_\text{Teil} \) der U-Wert des Bauteils ist.

Der gesamte Wärmeverlust pro Kelvin des Gebäudes ergibt sich als Summe der Wärmeverluste aller Bauteile:
\[
Q_{\text{Verlust}} = Q_{\text{Wand}} + Q_{\text{Boden}} + Q_{\text{Dach}} + Q_{\text{Fenster}} + Q_{\text{Tür}}
\]

Es handelt sich damit um eine vereinfachte Berechnung, die die Wärmebrücken und die thermische Speicherkapazität des Gebäudes nicht berücksichtigt.

\subsection{Maximale Heizlast und Temperaturdifferenz}
Die maximale Heizlast wird basierend auf der Normaußentemperatur \( T_{\text{außen}} \) und der gewünschten Raumtemperatur \( T_{\text{innen}} \) berechnet. Die maximale Temperaturdifferenz ist:
\[
\Delta T_{\text{max}} = T_{\text{innen}} - T_{\text{außen}}
\]
Die maximale Heizlast des Gebäudes ergibt sich aus den Wärmeverlusten pro Kelvin multipliziert mit der maximalen Temperaturdifferenz:
\[
Q_{\text{max}} = Q_{\text{Verlust}} \times \Delta T_{\text{max}}
\]

\subsection{Berechnung des Jahresheizbedarfs}
Der jährliche Heizwärmebedarf wird basierend auf den stündlichen Außentemperaturen berechnet. Die Heizlast für jede Stunde wird berechnet als:
\[
Q_{\text{Heizung, stunde}} = \max(m \times T_{\text{außen}} + b, 0)
\]
wobei \( m \) und \( b \) durch lineare Regression bestimmt werden. Die Summe der Heizlasten aller Stunden ergibt den jährlichen Heizwärmebedarf.

\subsection{Warmwasserbedarf}
Der jährliche Warmwasserbedarf wird als Konstante pro Quadratmeter Gebäudefläche und pro Stockwerk berechnet:
\[
Q_{\text{WW}} = \text{WW\_Bedarf\_pro\_m2} \times A_{\text{Boden}} \times \text{Stockwerke}
\]

\section{Wirtschaftliche Analyse}
Die Klasse \texttt{SanierungsAnalyse} führt eine wirtschaftliche Analyse der Sanierung durch. Es werden verschiedene Indikatoren wie Amortisationszeit, Net Present Value (NPV) und Return on Investment (ROI) berechnet.

\subsection{Kosteneinsparungen durch Sanierung}
Die jährlichen Energiekosteneinsparungen durch die Sanierung werden berechnet als Differenz zwischen den Energiekosten vor und nach der Sanierung:
\[
\Delta \text{Kosten} = Q_{\text{ref}} \times P_{\text{ref}} - Q_{\text{san}} \times P_{\text{san}}
\]
wobei \( Q_{\text{ref}} \) und \( Q_{\text{san}} \) die Wärmebedarfe vor und nach der Sanierung sind, und \( P_{\text{ref}} \) und \( P_{\text{san}} \) die Energiepreise vor und nach der Sanierung.

\subsection{Amortisationszeit}
Die Amortisationszeit gibt an, nach wie vielen Jahren die Investitionskosten durch die jährlichen Kosteneinsparungen gedeckt sind. Sie wird berechnet als:
\[
\text{Amortisationszeit} = \frac{\text{Investitionskosten}}{\Delta \text{Kosten}}
\]

\subsection{Net Present Value (NPV)}
Der NPV gibt den heutigen Wert zukünftiger Cashflows (Kosteneinsparungen) an, abzüglich der Investitionskosten. Er wird mit folgender Formel berechnet:
\[
\text{NPV} = \sum_{t=1}^{T} \frac{\Delta \text{Kosten}}{(1 + r)^t} - \text{Investitionskosten}
\]
wobei \( r \) der Diskontsatz und \( T \) die Anzahl der Jahre ist.

\subsection{Return on Investment (ROI)}
Der ROI gibt das Verhältnis zwischen den Kosteneinsparungen und den Investitionskosten an:
\[
\text{ROI} = \frac{\Delta \text{Kosten} \times T - \text{Investitionskosten}}{\text{Investitionskosten}}
\]

\section{Lebenszykluskostenanalyse (LCCA)}
Die Lebenszykluskostenanalyse berücksichtigt alle Kosten über die gesamte Lebensdauer der Sanierung, einschließlich Investitions-, Betriebs- und Instandhaltungskosten sowie den Restwert:
\[
\text{LCCA} = \text{NPV der Cashflows} + \text{Restwert}
\]

\section{Zusammenfassung}
Die \texttt{Sanierungsanalyse}-Berechnungsmethode ermöglicht es, den energetischen Zustand eines Gebäudes zu analysieren und die Kosten-Nutzen-Relation einer Sanierung zu bewerten. Die Berechnungen umfassen den Heiz- und Warmwasserbedarf des Gebäudes sowie wirtschaftliche Kennzahlen wie Amortisationszeit, NPV und ROI. Diese Methoden bieten eine fundierte Grundlage für Entscheidungen im Rahmen von Sanierungsprojekten.
