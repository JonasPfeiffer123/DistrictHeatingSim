\section{GasBoiler Klasse}
Die \texttt{GasBoiler}-Klasse repräsentiert ein Gaskesselsystem, das dazu dient, die Leistung, Kosten und Emissionen eines Gaskessels in einem Heizsystem zu berechnen und zu simulieren. Die Klasse umfasst zentrale ökonomische, betriebliche und ökologische Faktoren und ermöglicht eine umfassende Analyse in Energiesystemen.

\subsection{Attribute}
\begin{itemize}
    \item \texttt{name (str)}: Name des Gaskesselsystems.
    \item \texttt{spez\_Investitionskosten (float)}: Spezifische Investitionskosten für den Gaskessel in €/kW.
    \item \texttt{Nutzungsgrad (float)}: Wirkungsgrad des Gaskessels, der typischerweise zwischen 0,8 und 1,0 liegt. Er repräsentiert das Verhältnis von nutzbarer Wärmeleistung zur gesamten zugeführten Energie.
    \item \texttt{Faktor\_Dimensionierung (float)}: Dimensionierungsfaktor, der eine eventuelle Überdimensionierung berücksichtigt.
    \item \texttt{Nutzungsdauer (int)}: Lebensdauer des Gaskessels in Jahren. Standardmäßig 20 Jahre.
    \item \texttt{f\_Inst (float)}: Installationsfaktor, der zusätzliche Kosten aufgrund von Installationskomplexität repräsentiert.
    \item \texttt{f\_W\_Insp (float)}: Inspektionsfaktor, der periodische Wartungs- und Inspektionskosten berücksichtigt.
    \item \texttt{Bedienaufwand (float)}: Betriebskosten in Form von Arbeitsaufwand.
    \item \texttt{co2\_factor\_fuel (float)}: CO$_2$-Emissionsfaktor für den Brennstoff (Erdgas), typischerweise in tCO$_2$/MWh.
    \item \texttt{primärenergiefaktor (float)}: Primärenergiefaktor für den Brennstoff, der die Menge an Primärenergie darstellt, die benötigt wird, um eine Einheit nutzbare Energie (MWh) zu erzeugen. Dieser Faktor berücksichtigt Energieverluste in der Brennstofflieferkette.
\end{itemize}

\subsection{Methoden}
Die \texttt{GasBoiler}-Klasse enthält mehrere Methoden, die die Auslegung und Berechnung eines Gaskessels im Detail beschreiben. Im Folgenden werden die mathematischen Grundlagen und Berechnungslogiken der wichtigsten Methoden erläutert.

\subsubsection{simulate\_operation(Last\_L, duration)}
Diese Methode simuliert den Betrieb des Gaskessels basierend auf einem gegebenen Lastprofil und der Betriebsdauer. Sie berechnet die Wärmeerzeugung des Kessels, den Brennstoffbedarf sowie die maximale Leistung. Die wichtigsten Schritte der Berechnung sind:

\begin{itemize}
    \item \textbf{Berechnung der Wärmeleistung:} Zunächst wird das gegebene Lastprofil \texttt{Last\_L} verwendet, um die stündliche Wärmeleistung in kW zu bestimmen. Da negative Lasten (falls vorhanden) keinen Sinn ergeben, wird die Funktion \texttt{np.maximum()} verwendet, um negative Werte auf 0 zu setzen:
    \[
    \texttt{Wärmeleistung\_kW} = \max(\texttt{Last\_L}, 0)
    \]
    
    \item \textbf{Berechnung der Wärmemenge:} Die Wärmemenge (\texttt{Wärmemenge\_Gaskessel}) wird über die Summe der stündlichen Wärmeleistung, multipliziert mit der Simulationsdauer \texttt{duration} in Stunden, berechnet:
    \[
    \texttt{Wärmemenge\_Gaskessel} = \sum_{t=1}^{n} \left( \frac{\texttt{Wärmeleistung\_kW}[t]}{1000} \right) \times \texttt{duration}
    \]
    Dabei wird die Wärmeleistung von kW in MWh umgerechnet (Faktor 1000).

    \item \textbf{Berechnung des Gasbedarfs:} Der Gasbedarf (\texttt{Gasbedarf}) wird aus der erzeugten Wärmemenge und dem Wirkungsgrad (\texttt{Nutzungsgrad}) des Gaskessels berechnet:
    \[
    \texttt{Gasbedarf} = \frac{\texttt{Wärmemenge\_Gaskessel}}{\texttt{Nutzungsgrad}}
    \]
    Der Wirkungsgrad berücksichtigt die Verluste, die bei der Umwandlung von Brennstoff in nutzbare Wärme entstehen.

    \item \textbf{Maximale Leistung:} Die maximale Leistung des Gaskessels (\texttt{P\_max}) wird basierend auf der maximalen Last im Profil und einem Dimensionierungsfaktor berechnet, der mögliche Überdimensionierungen des Kessels berücksichtigt:
    \[
    \texttt{P\_max} = \max(\texttt{Last\_L}) \times \texttt{Faktor\_Dimensionierung}
    \]
\end{itemize}

\subsubsection{calculate\_heat\_generation\_cost(Brennstoffkosten, q, r, T, BEW, stundensatz)}
Diese Methode berechnet die Wärmegestehungskosten (\textbf{WGK}). Diese beinhalten sowohl Investitions- als auch Betriebskosten, um die tatsächlichen Kosten der Wärmeerzeugung pro MWh zu ermitteln. Die Berechnung erfolgt in mehreren Schritten:

\begin{itemize}
    \item \textbf{Berechnung der Investitionskosten:} Die spezifischen Investitionskosten (\texttt{spez\_Investitionskosten}) werden mit der maximalen Leistung des Kessels (\texttt{P\_max}) multipliziert, um die gesamten Investitionskosten (\texttt{Investitionskosten}) zu erhalten:
    \[
    \texttt{Investitionskosten} = \texttt{spez\_Investitionskosten} \times \texttt{P\_max}
    \]
    
    \item \textbf{Annuität:} Um die jährlichen Kapitalrückzahlungen zu berechnen, wird der Annuitätenfaktor verwendet, der sowohl die Kapitalrückzahlung über die Lebensdauer des Kessels als auch Installations- und Wartungskosten berücksichtigt. Der Annuitätenfaktor \texttt{A\_N} berechnet sich mit der Funktion \texttt{annuität}, die auf den Kapitalrückgewinnungsfaktor (\texttt{q}), die Lebensdauer (\texttt{T}), und die Installations- und Wartungskosten (\texttt{f\_Inst} und \texttt{f\_W\_Insp}) eingeht:
    \[
    \texttt{A\_N} = \text{annuität}(\texttt{Investitionskosten}, \texttt{Nutzungsdauer}, \texttt{f\_Inst}, \texttt{f\_W\_Insp}, \texttt{Bedienaufwand}, q, r, T)
    \]

    \item \textbf{Berechnung der Wärmegestehungskosten:} Die jährlichen Gesamtkosten werden durch die erzeugte Wärmemenge geteilt, um die spezifischen Wärmegestehungskosten (\texttt{WGK\_GK}) zu erhalten:
    \[
    \texttt{WGK\_GK} = \frac{\texttt{A\_N}}{\texttt{Wärmemenge\_Gaskessel}}
    \]
    Diese Kosten beinhalten die Investitionskosten, Betriebskosten und den Gaspreis.
\end{itemize}

\subsubsection{calculate(Gaspreis, q, r, T, BEW, stundensatz, duration, Last\_L, general\_results)}
Diese Methode führt eine vollständige Berechnung der Systemleistung und Kostenanalyse durch. Sie kombiniert die oben beschriebenen Schritte und berechnet die wichtigsten ökonomischen und ökologischen Kennzahlen:

\begin{itemize}
    \item \textbf{Berechnung der Wärmemenge und des Gasbedarfs:} Die Methode ruft \texttt{Gaskessel()} auf, um die Wärmemenge und den Gasbedarf zu berechnen.
    
    \item \textbf{Berechnung der CO$_2$-Emissionen:} Die CO$_2$-Emissionen werden auf Basis des Gasverbrauchs und des spezifischen CO$_2$-Faktors für Erdgas (\texttt{co2\_factor\_fuel}) berechnet:
    \[
    \texttt{co2\_emissions} = \texttt{Gasbedarf} \times \texttt{co2\_factor\_fuel}
    \]
    Um die spezifischen CO$_2$-Emissionen pro erzeugte Wärmeeinheit (in tCO$_2$/MWh) zu ermitteln, werden die gesamten CO$_2$-Emissionen durch die Wärmemenge geteilt:
    \[
    \texttt{spec\_co2\_total} = \frac{\texttt{co2\_emissions}}{\texttt{Wärmemenge\_Gaskessel}}
    \]
    
    \item \textbf{Primärenergieverbrauch:} Der Primärenergieverbrauch wird durch Multiplikation des Gasverbrauchs mit dem Primärenergiefaktor (\texttt{primärenergiefaktor}) berechnet:
    \[
    \texttt{primärenergie} = \texttt{Gasbedarf} \times \texttt{primärenergiefaktor}
    \]
    
    \item \textbf{Ergebnisse:} Am Ende werden die berechneten Werte in einem Wörterbuch (\texttt{results}) zurückgegeben, das die Wärmemenge, die zeitlich aufgelöste Wärmeleistung (\texttt{Wärmeleistung\_L}), den Brennstoffbedarf, die gewichteten Durchschnittskosten (\texttt{WGK}), die spezifischen CO$_2$-Emissionen und den Primärenergieverbrauch enthält.
\end{itemize}

Die vollständige Berechnungsmethode ermöglicht die Simulation der Leistung eines Gaskessels über einen bestimmten Zeitraum und liefert umfassende ökonomische und ökologische Kennzahlen, die für eine energetische Bewertung entscheidend sind.

\subsection{Ökonomische und ökologische Überlegungen}
Die \texttt{GasBoiler}-Klasse wurde entwickelt, um sowohl die ökonomischen als auch die ökologischen Auswirkungen eines Gaskesselsystems zu simulieren. Die \textbf{Wärmegestehungskosten (WGK)} berücksichtigen sowohl Investitionskosten als auch Betriebskosten, einschließlich Brennstoffpreise, Arbeitskosten und Wartung. Zudem werden die \textbf{CO$_2$-Emissionen} des Systems basierend auf dem Brennstoffverbrauch und dem spezifischen CO$_2$-Faktor für Erdgas berechnet, um eine Analyse des ökologischen Fußabdrucks des Systems zu ermöglichen. Der \textbf{Primärenergieverbrauch} wird ebenfalls berechnet, um Einblicke in die Gesamtenergieeffizienz und Nachhaltigkeit des Systems zu geben.

\subsection{Nutzungsbeispiel}
Das folgende Beispiel zeigt, wie die \texttt{GasBoiler}-Klasse initialisiert und verwendet werden kann:

\begin{verbatim}
gas_boiler = GasBoiler(
    name="Gasheizkessel",
    spez_Investitionskosten=35,  # €/kW
    Nutzungsgrad=0.92,  # 92% Effizienz
    Faktor_Dimensionierung=1.1  # Leichte Überdimensionierung
)

results = gas_boiler.calculate(
    Gaspreis=30,  # €/MWh
    q=0.03, r=0.02, T=20, BEW=1, 
    stundensatz=50, 
    duration=1, 
    Last_L=load_profile, 
    general_results={'Restlast_L': residual_load}
)
\end{verbatim}

In diesem Beispiel wird der Gaskessel mit einem Wirkungsgrad von 92\% und einer leichten Überdimensionierung dimensioniert. Die Berechnungsmethode schätzt die Wärmeerzeugung, den Gasbedarf, die CO$_2$-Emissionen und die gewichteten Durchschnittskosten der Wärmeerzeugung basierend auf einem Lastprofil und allgemeinen Systemparametern.
