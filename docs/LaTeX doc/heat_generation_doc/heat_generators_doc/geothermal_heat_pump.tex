\section{Geothermal Klasse}
Die \texttt{Geothermal}-Klasse modelliert ein geothermisches Wärmepumpensystem und erbt von der \texttt{HeatPump}-Basis-Klasse. Sie enthält Methoden zur Simulation des geothermischen Wärmeentzugsprozesses und zur Berechnung verschiedener ökonomischer und ökologischer Kennzahlen.

\subsection{Attribute}
\begin{itemize}
    \item \texttt{Fläche (float)}: Verfügbare Fläche für die geothermische Installation in Quadratmetern.
    \item \texttt{Bohrtiefe (float)}: Bohrtiefe der geothermischen Sonden in Metern.
    \item \texttt{Temperatur\_Geothermie (float)}: Temperatur der geothermischen Quelle in Grad Celsius.
    \item \texttt{spez\_Bohrkosten (float)}: Spezifische Bohrkosten pro Meter. Standardwert: 100 €/m.
    \item \texttt{spez\_Entzugsleistung (float)}: Spezifische Entzugsleistung pro Meter. Standardwert: 50 W/m.
    \item \texttt{Vollbenutzungsstunden (float)}: Vollbenutzungsstunden pro Jahr. Standardwert: 2400 Stunden.
    \item \texttt{Abstand\_Sonden (float)}: Abstand zwischen den Sonden in Metern. Standardwert: 10 m.
    \item \texttt{min\_Teillast (float)}: Minimale Teillast als Anteil der Nennlast. Standardwert: 0,2.
    \item \texttt{co2\_factor\_electricity (float)}: CO$_2$-Emissionsfaktor für Stromverbrauch, in tCO$_2$/MWh. Standardwert: 0,4 tCO$_2$/MWh.
    \item \texttt{primärenergiefaktor (float)}: Primärenergiefaktor für den Stromverbrauch. Standardwert: 2,4.
\end{itemize}

\subsection{Methoden}
\begin{itemize}
    \item \texttt{Geothermie(Last\_L, VLT\_L, COP\_data, duration)}: Simuliert den geothermischen Wärmeentzugsprozess und berechnet die erzeugte Wärmemenge, den Strombedarf und weitere Leistungskennzahlen.
    \begin{itemize}
        \item \textbf{Last\_L (array-like)}: Lastprofil in kW.
        \item \textbf{VLT\_L (array-like)}: Vorlauftemperaturen in Grad Celsius.
        \item \textbf{COP\_data (array-like)}: Daten zur Leistungszahl (COP) zur Interpolation.
        \item \textbf{duration (float)}: Dauer des Zeitschritts in Stunden.
    \end{itemize}
    Diese Methode berechnet den geothermischen Wärmeertrag auf Basis der Quelltemperatur und der spezifischen Entzugsleistung pro Meter. Die Entzugsleistung wird als:
    \[
    \texttt{Entzugsleistung} = \texttt{Bohrtiefe} \times \texttt{spez\_Entzugsleistung} \times \texttt{Anzahl\_Sonden}
    \]
    berechnet, wobei die Anzahl der Sonden von der verfügbaren Fläche und dem Abstand der Sonden abhängt.
    
    \item \texttt{calculate(VLT\_L, COP\_data, Strompreis, q, r, T, BEW, stundensatz, duration, general\_results)}: Berechnet die ökonomischen und ökologischen Kennzahlen für das geothermische Wärmepumpensystem.
    \begin{itemize}
        \item \textbf{VLT\_L (array-like)}: Vorlauftemperaturen in Grad Celsius.
        \item \textbf{COP\_data (array-like)}: COP-Daten zur Leistungsberechnung.
        \item \textbf{Strompreis (float)}: Strompreis in €/MWh.
        \item \textbf{q (float)}: Kapitalrückgewinnungsfaktor.
        \item \textbf{r (float)}: Preissteigerungsfaktor.
        \item \textbf{T (int)}: Betrachtungszeitraum in Jahren.
        \item \textbf{BEW (float)}: Abzinsungsfaktor für Betriebskosten.
        \item \textbf{stundensatz (float)}: Stundensatz in €/Stunde.
        \item \textbf{duration (float)}: Dauer jedes Simulationsschritts in Stunden.
        \item \textbf{general\_results (dict)}: Allgemeine Ergebnisse, inklusive Lastprofil.
    \end{itemize}
    Diese Methode berechnet die gewichteten Durchschnittskosten der Wärmeerzeugung (WGK) und die CO$_2$-Emissionen basierend auf dem Stromverbrauch. Die spezifischen CO$_2$-Emissionen werden wie folgt berechnet:
    \[
    \texttt{spec\_co2\_total} = \frac{\texttt{co2\_emissions}}{\texttt{Wärmemenge\_Geothermie}} \, \text{tCO$_2$/MWh}
    \]

    \item \texttt{to\_dict()}: Wandelt die Objektattribute in ein Wörterbuch um.
    
    \item \texttt{from\_dict(data)}: Erstellt ein Objekt aus einem Wörterbuch von Attributen.
\end{itemize}

\subsection{Ökonomische und ökologische Überlegungen}
Die \texttt{Geothermal}-Klasse berechnet die gewichteten Durchschnittskosten der Wärmeerzeugung (WGK), welche die Kosten für Bohrung, Installation, Betrieb und Stromverbrauch berücksichtigen. Sie berechnet außerdem die spezifischen CO$_2$-Emissionen basierend auf dem Stromverbrauch sowie den Primärenergieverbrauch unter Verwendung eines Primärenergiefaktors.

\subsection{Nutzungsbeispiel}
Das folgende Beispiel zeigt, wie die \texttt{Geothermal}-Klasse initialisiert und verwendet wird, um die Leistung eines geothermischen Systems zu berechnen:

\begin{verbatim}
geothermal_system = Geothermal(
    name="Geothermal Heat Pump",
    Fläche=500,  # m²
    Bohrtiefe=150,  # m
    Temperatur_Geothermie=10,  # °C
    spez_Bohrkosten=120,  # €/m
    spez_Entzugsleistung=55  # W/m
)
results = geothermal_system.calculate(
    VLT_L=temperature_profile, 
    COP_data=cop_profile, 
    Strompreis=100,  # €/MWh
    q=0.04, r=0.02, T=20, 
    BEW=0.9, 
    stundensatz=50, 
    duration=1, 
    general_results=load_data
)
\end{verbatim}
In diesem Beispiel wird ein geothermisches System mit einer Fläche von 500 m² und einer Bohrtiefe von 150 m simuliert. Die Leistung und die ökonomischen Kennzahlen des Systems werden anhand der eingegebenen Daten berechnet.
